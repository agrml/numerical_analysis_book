% После процента идут комментарии

% Класс документа
\documentclass[12pt, a4paper, twoside, openright]{report}
    % Размеры полей
    \usepackage[left=15mm, right=15mm, bottom=25mm, top=25mm, papersize={165mm, 240mm}]{geometry}
    
    % Кодировка файла
    \usepackage[T2A]{fontenc}
    \usepackage[utf8]{inputenc}
    % Переносы для русского языка
    \usepackage[russian]{babel}
    
    
    \usepackage{graphicx}
    \graphicspath{{image/}}
    \usepackage{tikz}
    \usepackage{epstopdf}
    \usepackage{hyperref}
    \usepackage{float}
    
    
    \hypersetup{
        colorlinks,
        breaklinks,
        linkcolor=[rgb]{0.07, 0.03, 0.56},
        citecolor=[rgb]{0.1, 0.45, 0.27},
        %colorlinks=true,% hyperlinks will be coloured
        urlcolor=[rgb]{0.07, 0.03, 0.56},
        %linkbordercolor=[rgb]{0.07, 0.03, 0.56}
    }
    
    \newcommand{\myfontsizes}{\fontsize{8}{14}\selectfont}
    \usepackage{fancyhdr}
    % ИЗМЕНЕНИЕ ШРИФТА В КОЛОНТИТУЛАХ
    \fancyhf{}
    \fancyhead[LE, RO]{\myfontsizes \thepage}
    \fancyhead[RE]{{\myfontsizes \leftmark}}
    \fancyhead[LO]{{\myfontsizes \rightmark}}
    \fancyfoot{}
    
    
    
    \makeatletter
    \let\ps@plain\ps@empty
    \makeatother
    
    % Отступ абзаца после заголовка
    \usepackage{indentfirst}
    
    %\usepackage{titlesec}
    \renewcommand{\thesection}{\S~\arabic{section}}
    % Добавить символ параграфа в нумерации section
    %\let\oldthesection\thesection \renewcommand\thesection{\S \oldthesection}
    
    %\renewcommand{\thechapter}{Глава~\roman{chapter}}
    %\let\oldthesection\thesection \renewcommand\thesection{\S \oldthesection}
    \usepackage{titletoc}%
    \titlecontents{chapter}% <section-type>
    [0pt]% <left>
    {\bfseries}% <above-code>
    {\chaptername\ \thecontentslabel }% <numbered-entry-format>
    {}% <numberless-entry-format>
    {\hfill\contentspage}% <filler-page-format>
    
    %\titleformat{\chapter}{\thispagestyle{myheadings}\centering\hyphenpenalty=10000\normalfont\huge\bfseries}{Глава~\roman{chapter} \thechapter. }{0pt}{\Huge}
    %\makeatother
    
    
    \usepackage[intlimits]{amsmath}
    % Нумерация в пределах параграфа
    \numberwithin{equation}{section}
    % Изменить формат нумерации формул с (1.2) на (2)
    \renewcommand{\theequation}{\arabic{equation}}
    
    \renewcommand{\thefigure}{\arabic{figure}}
    
    \usepackage{cmap}
    % Разные математические шрифты
    % Например, для обозначения множества вещественных чисел
    \usepackage{amsfonts}
    % Математические символы
    \usepackage{amssymb}
    % Для изменения символов нумерации
    \usepackage{enumerate}
    % Для нумерованных списков
    \usepackage{enumitem}
    % Декартов крест
    \DeclareFontFamily{U}{mathx}{\hyphenchar\font45}
    \DeclareFontShape{U}{mathx}{m}{n}{
          <5> <6> <7> <8> <9> <10>
          <10.95> <12> <14.4> <17.28> <20.74> <24.88>
          mathx10
          }{}
    \DeclareSymbolFont{mathx}{U}{mathx}{m}{n}
    \DeclareMathSymbol{\bigtimes}{1}{mathx}{"91}
    
    \usepackage{array}
    
    % Теоремы
    \usepackage{amsthm}
    \newtheorem*{statement}{Утверждение}
    \newtheorem*{problem*}{Задача}
    \newtheorem{problem}{Задача}
    \newtheorem*{definition}{Определение}
    
    \theoremstyle{definition}
    \newtheorem{example}{Пример}[section]
    \renewcommand{\theexample}{\arabic{example}}
    \newtheorem*{example*}{Пример}
    
    \theoremstyle{plain}
    \newtheorem{theorem}{Теорема}[section]
    \renewcommand{\thetheorem}{\arabic{theorem}}
    \newtheorem*{theorem*}{Теорема}
    
    \newtheorem{new_note}{Замечание}[section]
    
    \newtheorem{note}{Замечание}[section]
    \renewcommand{\thenote}{\arabic{note}}
    
    
    
    \newtheorem{lemma}{Лемма}[section]
    \renewcommand{\thelemma}{\arabic{lemma}}
    \newtheorem*{lemma*}{Лемма}
    
    \newtheorem*{note*}{Замечание}
    
    \newtheorem{formulation}{Постановка задачи}
    
    \newtheorem*{formulation*}{Постановка задачи}
    
    \newtheorem{cons}{Следствие}[section]
    \renewcommand{\thecons}{\arabic{cons}}
    \newtheorem*{cons*}{Следствие}
    
    \makeatletter \renewenvironment{proof}[1][\proofname] {%
    \par\pushQED{\qed}\normalfont\topsep6\p@\@plus6\p@\relax\trivlist\item%
    [\hskip\labelsep\bfseries#1\@addpunct{.}]\ignorespaces}%
    {\popQED\endtrivlist\@endpefalse} \makeatother
    
    \makeatletter \newenvironment{solution}[1][Решение] {%
    \par\pushQED{\qed}\normalfont\topsep6\p@\@plus6\p@\relax\trivlist\item%
    [\hskip\labelsep\bfseries#1\@addpunct{.}]\ignorespaces}%
    {\popQED\endtrivlist\@endpefalse} \makeatother
    
    % Новая строка после названия пункта
    \makeatletter
    \renewcommand\paragraph{%
       \@startsection{paragraph}{4}{0mm}%
          {-\baselineskip}%
          {.5\baselineskip}%
          {\normalfont\large\bfseries}}
    \makeatother
    
    \makeatletter
    \DeclareRobustCommand*{\bfseries}{%
      \not@math@alphabet\bfseries\mathbf
      \fontseries\bfdefault\selectfont
      \boldmath
    }
    \makeatother
    
    % Для систем уравнений
    \usepackage{cases}
    
    \makeatletter
    \newenvironment{sqcases}{%
      \matrix@check\sqcases\env@sqcases
    }{%
      \endarray\right.%
    }
    \def\env@sqcases{%
      \let\@ifnextchar\new@ifnextchar
      \left\lbrack
      \def\arraystretch{1.2}%
      \array{@{}l@{\quad}l@{}}%
    }
    \makeatother
    
    \usepackage{mathtools}
    % Норма вектора
    \DeclarePairedDelimiter\norm{\lVert}{\rVert}
    % Целая часть сверху
    \DeclarePairedDelimiter{\ceil}{\lceil}{\rceil}
    % Оператор sgn
    \DeclareMathOperator{\sgn}{sgn}
    % Оператор diag
    \DeclareMathOperator{\diag}{diag}
    
    \renewcommand\labelitemi{---}
    % \fancyhead{}
    % \fancyhead[LE,RO]{\thepage}
    % \fancyhead[RE]{\leftmark}
    % \fancyhead[LO]{\rightmark}
    
    \setcounter{secnumdepth}{1}
    \setcounter{tocdepth}{1}
    \newcommand{\bigO}[1]{\ensuremath{\operatorname{O}\bigl(#1\bigr)}}
    \newcommand{\largeO}{\text{\LARGE{$\operatorname{O}$}}}
    
    \DeclareMathAlphabet{\mathbfit}{OML}{cmm}{b}{it}
    \newcommand{\vfunc}[1]{\mathbfit{#1}}
    
    \setlength{\headheight}{16pt}
    \bibliographystyle{utf8gost705u}  %% стилевой файл для оформления по ГОСТу
    
    \makeatletter
    \def\@biblabel#1{#1. }
    \makeatother
    
    %\makeatletter
    %\renewcommand\l@chapter[2]{%
    %  \ifnum \c@tocdepth >\m@ne
    %    \addpenalty{-\@highpenalty}%
    %    \vskip 1.0em \@plus\p@
    %    \setlength\@tempdima{1.5em}%
    %    \begingroup
    %      \parindent \z@ \rightskip \@pnumwidth
    %      \parfillskip -\@pnumwidth
    %      \leavevmode \bfseries
    %      \advance\leftskip\@tempdima
    %      \hskip -\leftskip
    %      #1\nobreak\hfil \nobreak\hb@xt@\@pnumwidth{\hss #2}\par
    %      \penalty\@highpenalty
    %    \endgroup
    %  \fi}
    %\makeatother
    
    \renewcommand{\thechapter}{\Roman{chapter} }
    \usepackage{imakeidx}
    \usepackage{multirow}
    \usepackage{tabularx}
    \usepackage{adjustbox}
    \usepackage{lscape}
    \makeindex
    \indexsetup{othercode=\footnotesize}
    %
    \begin{document}
    %
    \pagestyle{empty}
    %
    \begin{center}
        \vspace*{-0.7in}
        \includegraphics[width=0.5\textwidth]{msu}\\
        {\scshape Московский государственный университет имени М.\,В. Ломоносова}\\
        Факультет вычислительной математики и кибернетики\\
        \vfill
        {\LARGE Лекции по курсу}\\
        \vspace{0.55cm}
        {\Huge\bfseries Численные методы}\\
    \end{center}
    %
    \vspace{1cm}
    %
    \begin{center}
        \textit{Лектор}\\
        Н.\,И. Ионкин\\
    \end{center}
    
    %
    \vfill
    %
    \begin{center}
        Москва, 2018
    \end{center}

    %
    % \newpage
% %
% \phantomsection
% \pagestyle{empty}


\newpage
\phantomsection
\pagestyle{empty}
\chapter*{Приложение B}
\addcontentsline{toc}{chapter}{Приложение B}
\vspace{0.5cm}

В каждой главе приводится строгое математическое доказательство эффективности рассмотренных методов. Данное приложение представляет собой Компактное изложение результатов и фактов глав пособия. Для каждой главы простым языком излагаются ее основные идеи и приводится таблица сводных характеристик рассмотренных в ней методов.

% starter for a character
\newpage
\phantomsection
\pagestyle{empty}
\vspace{0.5cm}
% \starter


\section*{Глава I. Численные методы линейной алгебры}

\subsection{Основные идеи главы}

В главе I пособия рассматриваются основные методы решения систем линейных уравнений, поиска собственных значений матриц и нахождения обратной матрицы. Для численного решения СЛАУ применяются прямые и итерационные методы.

Прямые методы: например, метод Гаусса и квадратного корня — это методы, которые позволяют получать численные решения, исходя из формул, точно (с поправкой на ошибки округления). Эффективность прямых методов оценивается по числу действий (как правило умножений и делений), необходимых для реализации метода.

Итерационные методы основаны на том, что задается начальное приближение и указывается закон, по которому задаются следующие приближения. Эффективность итерационных методов оценивается числом итераций, необходимых для достижения заданной точности \begin{math} \epsilon \end{math}. Ясно, что чем меньшее число итераций потребуется, тем более эффективен метод.

Нахождение собственных значений матрицы, как правило, неразрешимая задача. В связи с этим в общем случае задача на поиск собственных значений решается численно с использованием итерационных методов. При численном решении задачи на собственные значения рассматриваются две проблемы: частичная проблема собственных значений, т.е. нахождение отдельных собственных значений и отвечающих им собственных векторов, и полной проблемы собственных значений, заключающася в нахождении всех собственных векторов матрицы.

Одно из важных понятий первой главы пособия — понятие обратной матрицы, которое активно используется не только в контексте прямого поиска решения, но и при исследовании на сходимость численных методов нахождения решений различных задач и оценке скорости их сходимости.

\subsection{Рассмотренные методы}
\begin{table}[ht]
\centering
\resizebox{\textwidth}{!}{
    \small
    \begin{tabular}{|p{3cm}|p{7cm}|p{7cm}|p{7cm}|}
    \hline
    \textbf{Название метода} & \textbf{Канонический вид} & \textbf{Условие сходимости в среднеквадратичной норме} & \textbf{Скорость сходимости} \\
    \hline
    \begin{tabular}[c]{@{}l@{}} Метод простой\\ итерации  \end{tabular}  &
    \begin{tabular}[c]{@{}l@{}}
    	\begin{math} \frac{x^{n + 1} - x^n}{\tau} + Ax^n = f, \tau > 0 \end{math} \\
    	\begin{math} \tau > 0, n \in Z_+ \end{math} \\
        \begin{math} x^0 \end{math} -- начальное приближение
    \end{tabular} &
    \begin{tabular}[c]{@{}l@{}}
    	\begin{math} A^T=A>0, \gamma=\max\limits_{1 \leq k \leq m}\lambda_k^A, 0<\tau<\frac{2}{\gamma} \end{math} \\
        \begin{math} x_0 \end{math} — любое
    \end{tabular} &
    \begin{tabular}[c]{@{}l@{}}
    	\begin{math} A=A^{*}>0,B=E \end{math} \\
        \begin{math} \gamma_1=\min\limits_{1 \leq k \leq m}\lambda_k^A, \gamma_2=\max\limits_{1 \leq k \leq m}\lambda_k^A \end{math} \\
        \begin{math} \xi=\frac{\gamma_1}{\gamma_2}, \rho=\frac{1-\xi}{1+\xi} \Rightarrow ||\nu^{n+1}|| \leq \rho||\nu^n|| \end{math}
    \end{tabular}
    \\
    \hline
    \end{tabular}
}
\end{table}
% starter for a character
\newpage
\phantomsection
\pagestyle{empty}
\vspace{0.5cm}
% \starter

\section*{Глава 2. Численное решение нелиней- ных уравнений и систем нелинейных уравнений}

    Задача интерполирования состоит в нахождении значений функции $f(x)$
    на всем отрезке $[a,b]$ по ее значениям в узловых точках. $\forall f(x)~\exists!$ интерполяционный полином степени $n$, построенный по $(n+1)$-му узлу. Легко сроится в виде полинома Лагранжа или Ньютона.

Если требуется дополнительно наложить условия на производные порядка $(a_k-1)$, $k = \overline{0,m}$ и $(sum_{k=0}^m a_k = n + 1) \Rightarrow \exists!$ интерполяционный полином Эрмита степени $n$, удовлетворяющий значениям функции и её производных в заданных точках.

В гильбертовом пространстве $L_2$ вещественных функций, интегрируемых с квадратом ($\int\limits_a^b{f^2(x)dx} < \infty$) со скалярным произведением $(f,g)=\int\limits_a^b{f(x)g(x)dx}$ и порождённой им нормой, для заданной ф-ии $f$ и системы $(n+1)$ линейно независимых ф-й $\{\varphi_i(x)\}_{i=0}^n$ существует их линейная комбинация, имеющая минимальное отклонение по норме от функции $f(x)$. Если система ортонормированная –выполняется неравенство Бесселя $\sum_{k=0}^n c_k^2 \leqslant \|f\|^2$, для ортонормированного базиса достигается равенство Парсеваля $\sum_{k=0}^\infty c_k^2 = \|f\|^2.$

Для линейного пространства функций $H$, заданных таблично (в узлах),  скалярным произведением $(f,g) = \sum\limits_{i=0}^N f_i g_i$ и порождённой им нормой приближение строится аналогичным образом.

\subsection{Рассмотренные методы}
% starter for a character
\newpage
\phantomsection
\pagestyle{empty}
\vspace{0.5cm}
% \starter


\section*{Глава III. Численное решение нелинейных уравнений и систем нелинейных уравнений}

\subsection{Основные идеи главы}

Для решения нелинейных уравнений и систем нелинейных уравнений применяются методы последовательного приближения решения. Их идея заключается в задании некоторого начального приближения и последующим отдалением от него в поиске решений.

В простейшем виде описанную идею реализует \textit{метод простой итерации}, где очередное приближение получается по формуле
\begin{math} x^{n + 1} = x^n + r(x^n)f(x^n) \end{math}.
Метод простой итерации не определяет функцию \begin{math} r() \end{math}. Оптимальным по скорости сходимости выбором является \begin{math} r(x) = -\frac{1}{f'(x^n)} \end{math}, что приводит к \textit{методу Ньютона}.
% Недостатком метода Ньютона является высокая вычислительная сложность, исходящая из необходимости вычисления производной на каждой итерации.
Выбор \begin{math} r(x) = - \frac{(x^n - x^{n - 1})f(x^n)}{f(x^n) - f(x^{n - 1})} \end{math} приводит к \textit{методу секущих}.

\subsection{Рассмотренные методы}
\begin{table}[H]
\begin{center}
\begin{tabular}{|c|c|c|}
\hline
\textbf{Название метода} & \textbf{Формула прерощения} & \textbf{Условие сходимости} \\
\hline
Метод & \begin{math} x^{n + 1} = x^n + r(x^n)f(x^n) \end{math} & \\
простой итерации &  & \\
\hline
Метод Ньютона & \begin{math} x^{n + 1} = x^n - \frac{f(x^n)}{f'(x^n)} \end{math} & \begin{math} \exists M>0: \end{math} \\
& & \begin{math} \frac{1}{2}\left|\left(\frac{f(x)f''(x)}{(f'(x))^2}\right)'\right| \leqslant M, \end{math} \\
& & \begin{math} ~~~x \in U_a(x^*), |x^0 -x^*| <= \frac{1}{M} \end{math} \\
\hline
Модифицированный & \begin{math} x^{n + 1} = x^n - \frac{f(x^n)}{f'(x^0)} \end{math} & -//-\\
метод Ньютона & & \\ 
\hline
Метод секущих & \begin{math} x^{n + 1} = - \frac{(x^n - x^{n - 1})f(x^n)}{f(x^n) - f(x^{n - 1})} \end{math} & \\
\hline
\end{tabular}
\end{center}
\end{table}
% starter for a character
\newpage
\phantomsection
\pagestyle{empty}
\vspace{0.5cm}
% \starter

\section*{Глава IV. Разностные методы решения задач математической физики}

\subsection{Основные идеи главы}

Задачи математической физики крайне редко решаются аналитически. Среди численных методов решения этих задач наиболее эффективными являются методы, основанные на построении разностных схем (т.н. разностные методы).

Для первой краевой задачи уравнения теплопроводности в главе IV рассмотрены явная и неявные разностные схемы. Доказано, что явная разностная схема условно устойчива и сходит при условии \begin{math} \frac{\tau}{h^2} \leq \frac{1}{2} \end{math} и имеет первый порядок точности по $\tau$ и второй по $h$. Априорная оценка доказана в норме $C$. Чисто неявная разностная схема является абсолютно устойчивой в норме $C$ и абсолютно сходящейся. Симметричная разностная схема (или схема Кранка-Никольсона) имеет второй порядок погрешности как по $\tau$, так и по $h$. Сходимость и устойчивость этой схемы доказывается в среднеквадратичной норме. Для первой краевой задачи уравнения теплопроводности можно построить разностные схемы заданного качества и имеющие второй порядок точности по $\tau$ и четвертый по $h$.

Изучена разностная схема на пятиточечном шаблоне для первой краевой задачи уравнения Пуассона (задача Дирихле) в прямоугольнике. Опираясь на принцип максимума и выбирая нужным образом мажоранту, получена априорная оценка устойчивости в норме $C$ по правой части уравнения.

\subsection{Рассмотренные методы}
\begin{table}[ht]
\centering
\resizebox{\textwidth}{!}{
    \small
    \begin{tabular}{|p{3cm}|p{7cm}|p{6.5cm}|p{6.5cm}|}
    
    \hline
    \textbf{Название метода} & \textbf{Описание метода} & \textbf{Погрешность аппроксимации} & \textbf{Сходимость численного решения к точному}
    \\
    
    \hline
    \begin{tabular}[c]{@{}l@{}} Явная \\ разностная \\ схема \end{tabular}  &
    \begin{tabular}[c]{@{}l@{}}
    	\begin{math} \frac{y_i^{n+1} - y_i^n}{\tau}=\frac{y_{i-1}^{n+1} - 2y_i^n + y_{i+1}^{n+1}}{h^2} + f(x_i,t_n),\end{math} \\
        \begin{math} (x_i,t_n)\in\omega_{\tau h}, \end{math} \\
        \begin{math} \begin{cases}
        	y_0^{n+1}=\mu_1(t_{n+1}), t_{n+1} \in \overline{\omega}_\tau \\ 
            y_N^{n+1}=\mu_1(t_{n+1}), t_{n+1} \in \overline{\omega}_\tau
        \end{cases}, \end{math} \\
        \begin{math} y_i^0=u_0(x_i), x_i \in \overline{\omega}_h \end{math}
    \end{tabular} &
    \begin{tabular}[c]{@{}l@{}}
    	\begin{math} \psi_i^n=\frac{u_{i-1}^{n+1} - 2u_i^n + u_{i+1}^{n+1}}{h^2}-\frac{u_i^{n+1} - u_i^n}{\tau}+f_i^n, \end{math} \\
        \begin{math} \psi_i^n=O(\tau+h^2) \end{math}
    \end{tabular} &
    \begin{tabular}[c]{@{}l@{}}
    	Сходится в норме C. \\
    	Необходимо и достаточно для \\
        сходимости и устойчивости: \\
    	\begin{math} \frac{\tau}{h^2}=\gamma\leq\frac{1}{2} \end{math}
    \end{tabular}
    \\
    
    \hline
    \end{tabular}
}
\end{table}
% starter for a character
\newpage
\phantomsection
\pagestyle{empty}
\vspace{0.5cm}
% \starter

\section*{Глава V. Методы решения обыкновенных дифференциальных уравнений и систем ОДУ}

\subsection{Основные идеи главы} 

Задача Коши для системы обыкновенных дифференциальных уравнений
\begin{equation}
%
    \label{Koshi_sys}
    %
    \begin{cases}
    %
        \dfrac{d\vfunc{u}}{dt} = \vfunc{f}(t, \vfunc{u}(t)), \quad t > 0, \\
        \vfunc{u}(0) = u_0,
    %
    \end{cases}
    %
\end{equation}
при условии существования и единственности решения (в прямоугольнике $\{|t| \leqslant a, | \vfunc{u}(t)-\vfunc{u}(0)| \leqslant b,\; a, b\in\mathbb{R}\}$ ф-я $f(t, u)$ непрерывна и удовлетвоняет условию Липшица по $u$ решается следующими способами: \textit{$m$-этапный метода Рунге--Кутта}, \textit{ $m$-шаговый разностный метод}.

При переносе методов на систему уравнений следует изучить понятие жёсткости и области устойчивости.
 
Первую краевую задачу для дифференциального уравнения второго порядка
\begin{equation}
    %
    \label{eq:2-ord-eq}
    %
    \dfrac{d}{dx}\left(k(x)\dfrac{du}{dx}\right) - q(x)u(x) + f(x) = 0, ~~~x\in(0,1)
\end{equation}
%
и краевым условиям первого рода при $x=0$, $x=1$
%
\begin{equation}
    %
    \label{eq:2-ord-eq-bounds}
    %
    u(0) = \mu_1,\;u(1) = \mu_2,
\end{equation}
%
решают интегро-интерполяционным методом (методом баланса) построения разностных схем.

\subsection{Рассмотренные методы}
% starter for a character
\newpage
\phantomsection
\pagestyle{empty}
\vspace{0.5cm}
% \starter

\section*{Глава VI. Приближённое вычисление определённых интегралов}

\subsection{Основные идеи главы} 
Одним из самых распространенных приемов приближенного вычисления определенных интегралов в вычислительной математике является подход, основанный на построении квадратурных формул. Смысл этих формул состоит в том, что определенный интеграл заменяется конечной суммой $\sum_{k=1}^{n}(c_k f(x_k))$, где $f(x_k)$ -- значение подынтегральной функции в узлах $x_k$, а выбор коэффициентов $c_k$ позволяет вычислить интеграл с требуемой точностью. Для рассмотренных в главе квадратурных форму исследуется погрешность вычисления как на частичном отрезке $[x_{i-1}, x_i], i = 1..n$, так и на всем промежутке интегрирования.

\begin{landscape}
\newpage
\subsection{Рассмотренные методы}


\end{landscape}

    
    
    \end{document}
    
    