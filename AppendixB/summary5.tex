% starter for a character
\newpage
\phantomsection
\pagestyle{empty}
\vspace{0.5cm}
% \starter

\section*{Глава V. Методы решения обыкновенных дифференциальных уравнениий и систем ОДУ}

\subsection{Основные идеи главы} 

Задача Коши для системы обыкновенных дифференциальных уравнений
\begin{equation}
%
    \label{Koshi_sys}
    %
    \begin{cases}
    %
        \dfrac{d\vfunc{u}}{dt} = \vfunc{f}(t, \vfunc{u}(t)), \quad t > 0, \\
        \vfunc{u}(0) = u_0,
    %
    \end{cases}
    %
\end{equation}
при условии существования и единственности решения (в прямоугольнике $\{|t| \leqslant a, | \vfunc{u}(t)-\vfunc{u}(0)| \leqslant b,\; a, b\in\mathbb{R}\}$ ф-я $f(t, u)$ непрерывна и удовлетвоняет условию Липшица по $u$ решается следующими способами: \textit{$m$-этапный метода Рунге--Кутта}, \textit{ $m$-шаговый разностный метод}.

При переносе методов на систему уравнений следует изучить понятие жёсткости и области устойчивости.
 
Первую краевую задачу для дифференциального уравнения второго порядка
\begin{equation}
    %
    \label{eq:2-ord-eq}
    %
    \dfrac{d}{dx}\left(k(x)\dfrac{du}{dx}\right) - q(x)u(x) + f(x) = 0, ~~~x\in(0,1)
\end{equation}
%
и краевым условиям первого рода при $x=0$, $x=1$
%
\begin{equation}
    %
    \label{eq:2-ord-eq-bounds}
    %
    u(0) = \mu_1,\;u(1) = \mu_2,
\end{equation}
%
решают интегро-интерполяционным методом (методом баланса) построения разностных схем.

\newpage

\subsection{Рассмотренные методы}
\begin{adjustbox}{angle=90}
\begin{adjustbox}{scale=1.4}
\centering
\resizebox{\textwidth}{!}{
    \small
    \begin{tabular}{|p{3.5cm}|p{6.7cm}|p{7.1cm}|p{6cm}|}
    
    \hline
    \textbf{Название метода} & \textbf{Описание метода} & \textbf{Погрешность формулы}\\
    
    \hline
    \begin{tabular}[c]{@{}l@{}} Метод Рунге-Кутта \end{tabular}  &
    \begin{tabular}[c]{@{}l@{}}
    	$\dfrac{y_{n+1} - y_n}{\tau} = \sigma_1 K_1 + \sigma_2 K_2 + \ldots + \sigma_m K_m$ \\
        $K_1 = f(t_n, y_n)$,\\
        $K_2 = f(t_n + a_2\tau, y_n + b_{21} \tau K_1)$, \\
        $K_3 = f(t_n + a_3\tau, y_n + b_{31}\tau K_1 + b_{32}\tau K_2)$, \\
        \dots \\
        $K_m = f(t_n+a_m\tau, y_n + b_{m1} \tau K_1 + $\\
        $+ b_{m2} \tau K_2 + \ldots + b_{m m -1} \tau K_{m-1})$
    \end{tabular} &
    \begin{tabular}[c]{@{}l@{}}
    	Однопараметрическое семейство\\ разностных схем \\
        $\dfrac{y_{n+1} - y_n}{\tau} = (1 - \sigma)f(t_n, y_n) +$\\
        $\sigma f(t_n + a\tau, y_n + a\tau f(t_n, y_n))$\\
        $\psi_n = O(\tau^2)$, если $a\tau=0.5$\\
        $z_n \leq M\tau^2, M > 0$ и не зависит от $\tau$
    \end{tabular}
    \\
    
    \hline
    \begin{tabular}[c]{@{}l@{}} Многошаговые \\разностные методы\end{tabular}  &
    \begin{tabular}[c]{@{}l@{}}
         $\sum_{k=0}^m \frac{a_k}{\tau} y_{n-k} = \sum_{k=0}^m b_k f_{n-k},$ \\
         $a_0 \not= 0, b_m \not= 0, n=m, m+1, \ldots$. \\
         $y_{n-1}, y_{n-2}, \ldots, y_{n-m}$ - заданы \\
         При $b=0$ метод называется явным, \\
         в противном случае неявным. \\
         Условие нормировки: $\sum_{k=0}^m b_k$
    \end{tabular} &
    \begin{tabular}[c]{@{}l@{}}
    	 Для достижения порядка аппроксимации\\
         $p$ должны выполняться \\
         следующие условия:\\
         $a_0=-\sum_{k=1}^m a_k$\\
         $b_0=1-\sum_{k=1}^m b_k$\\
         $\sum_{k=0}^m k^{l-1}(a_kk+lb_k)=0,l=1,2,\ldots,p$
    \end{tabular}
    \\
    \hline
    \end{tabular}
}
\end{adjustbox}
\end{adjustbox}