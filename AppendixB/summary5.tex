% starter for a character
\newpage
\phantomsection
\pagestyle{empty}
\vspace{0.5cm}
% \starter

\section*{Глава 5. Методы решения обыкновенных дифференциальных уравнений и систем ОДУ}

Задача Коши для системы обыкновенных дифференциальных уравнений
\begin{equation}
%
    \label{Koshi_sys}
    %
    \begin{cases}
    %
        \dfrac{d\vfunc{u}}{dt} = \vfunc{f}(t, \vfunc{u}(t)), \quad t > 0, \\
        \vfunc{u}(0) = u_0,
    %
    \end{cases}
    %
\end{equation}
при условии существования и единственности решения (в прямоугольнике $\{|t| \leqslant a, | \vfunc{u}(t)-\vfunc{u}(0)| \leqslant b,\; a, b\in\mathbb{R}\}$ ф-я $f(t, u)$ непрерывна и удовлетвоняет условию Липшица по $u$ решается следующими способами: \textit{$m$-этапный метода Рунге--Кутта}, \textit{ $m$-шаговый разностный метод}.

При переносе методов на систему уравнений следует изучить понятие жёсткости и области устойчивости.
 
Первую краевую задачу для дифференциального уравнения второго порядка
\begin{equation}
    %
    \label{eq:2-ord-eq}
    %
    \dfrac{d}{dx}\left(k(x)\dfrac{du}{dx}\right) - q(x)u(x) + f(x) = 0, ~~~x\in(0,1)
\end{equation}
%
и краевым условиям первого рода при $x=0$, $x=1$
%
\begin{equation}
    %
    \label{eq:2-ord-eq-bounds}
    %
    u(0) = \mu_1,\;u(1) = \mu_2,
\end{equation}
%
решают интегро-интерполяционным методом (методом баланса) построения разностных схем.

\subsection{Рассмотренные методы}