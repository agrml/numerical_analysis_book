% starter for a character
\newpage
\phantomsection
\pagestyle{empty}
\vspace{0.5cm}
% \starter

\section*{Глава IV. Разностные методы решения задач математической физики}
Численные методы расширяют класс решаемых задач математической физики, позволяя находить приближенное решение для дифференциальных задач, для которых не существует аналитических методов решения. В главе IV пособия рассматриваются разностные схемы.

Разностные схемы делятся на явные и неявные. В главе рассматривается применение схем на примере классических задач математической физики, а также необходимые допущения и основные понятия теории разностных схем: аппроксимация, погрешность, устойчивость, сходимость.

\subsection{Рассмотренные методы}
