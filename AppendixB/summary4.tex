% starter for a character
\newpage
\phantomsection
\pagestyle{empty}
\vspace{0.5cm}
% \starter

\section*{Глава IV. Разностные методы решения задач математической физики}

\subsection{Основные идеи главы}

Задачи математической физики крайне редко решаются аналитически. Среди численных методов решения этих задач наиболее эффективными являются методы, основанные на построении разностных схем (т.н. разностные методы).

Для первой краевой задачи уравнения теплопроводности в главе IV рассмотрены явная и неявные разностные схемы. Доказано, что явная разностная схема условно устойчива и сходит при условии \begin{math} \frac{\tau}{h^2} \leq \frac{1}{2} \end{math} и имеет первый порядок точности по $\tau$ и второй по $h$. Априорная оценка доказана в норме $C$. Чисто неявная разностная схема является абсолютно устойчивой в норме $C$ и абсолютно сходящейся. Симметричная разностная схема (или схема Кранка-Никольсона) имеет второй порядок погрешности как по $\tau$, так и по $h$. Сходимость и устойчивость этой схемы доказывается в среднеквадратичной норме. Для первой краевой задачи уравнения теплопроводности можно построить разностные схемы заданного качества и имеющие второй порядок точности по $\tau$ и четвертый по $h$.

Изучена разностная схема на пятиточечном шаблоне для первой краевой задачи уравнения Пуассона (задача Дирихле) в прямоугольнике. Опираясь на принцип максимума и выбирая нужным образом мажоранту, получена априорная оценка устойчивости в норме $C$ по правой части уравнения.

\subsection{Рассмотренные методы}
\begin{table}[ht]
\centering
\resizebox{\textwidth}{!}{
    \small
    \begin{tabular}{|p{3cm}|p{7cm}|p{6.5cm}|p{6.5cm}|}
    
    \hline
    \textbf{Название метода} & \textbf{Описание метода} & \textbf{Погрешность аппроксимации} & \textbf{Сходимость численного решения к точному}
    \\
    
    \hline
    \begin{tabular}[c]{@{}l@{}} Явная \\ разностная \\ схема \end{tabular}  &
    \begin{tabular}[c]{@{}l@{}}
    	\begin{math} \frac{y_i^{n+1} - y_i^n}{\tau}=\frac{y_{i-1}^{n+1} - 2y_i^n + y_{i+1}^{n+1}}{h^2} + f(x_i,t_n),\end{math} \\
        \begin{math} (x_i,t_n)\in\omega_{\tau h}, \end{math} \\
        \begin{math} \begin{cases}
        	y_0^{n+1}=\mu_1(t_{n+1}), t_{n+1} \in \overline{\omega}_\tau \\ 
            y_N^{n+1}=\mu_1(t_{n+1}), t_{n+1} \in \overline{\omega}_\tau
        \end{cases}, \end{math} \\
        \begin{math} y_i^0=u_0(x_i), x_i \in \overline{\omega}_h \end{math}
    \end{tabular} &
    \begin{tabular}[c]{@{}l@{}}
    	\begin{math} \psi_i^n=\frac{u_{i-1}^{n+1} - 2u_i^n + u_{i+1}^{n+1}}{h^2}-\frac{u_i^{n+1} - u_i^n}{\tau}+f_i^n, \end{math} \\
        \begin{math} \psi_i^n=O(\tau+h^2) \end{math}
    \end{tabular} &
    \begin{tabular}[c]{@{}l@{}}
    	Сходится в норме C. \\
    	Необходимо и достаточно для \\
        сходимости и устойчивости: \\
    	\begin{math} \frac{\tau}{h^2}=\gamma\leq\frac{1}{2} \end{math}
    \end{tabular}
    \\
    
    \hline
    \end{tabular}
}
\end{table}