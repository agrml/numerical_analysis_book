% starter for a character
\newpage
\phantomsection
\pagestyle{empty}
\vspace{0.5cm}
% \starter

\section*{Глава 2. Численное решение нелиней- ных уравнений и систем нелинейных уравнений}

    Задача интерполирования состоит в нахождении значений функции $f(x)$
    на всем отрезке $[a,b]$ по ее значениям в узловых точках. $\forall f(x)~\exists!$ интерполяционный полином степени $n$, построенный по $(n+1)$-му узлу. Легко сроится в виде полинома Лагранжа или Ньютона.

Если требуется дополнительно наложить условия на производные порядка $(a_k-1)$, $k = \overline{0,m}$ и $(sum_{k=0}^m a_k = n + 1) \Rightarrow \exists!$ интерполяционный полином Эрмита степени $n$, удовлетворяющий значениям функции и её производных в заданных точках.

В гильбертовом пространстве $L_2$ вещественных функций, интегрируемых с квадратом ($\int\limits_a^b{f^2(x)dx} < \infty$) со скалярным произведением $(f,g)=\int\limits_a^b{f(x)g(x)dx}$ и порождённой им нормой, для заданной ф-ии $f$ и системы $(n+1)$ линейно независимых ф-й $\{\varphi_i(x)\}_{i=0}^n$ существует их линейная комбинация, имеющая минимальное отклонение по норме от функции $f(x)$. Если система ортонормированная –выполняется неравенство Бесселя $\sum_{k=0}^n c_k^2 \leqslant \|f\|^2$, для ортонормированного базиса достигается равенство Парсеваля $\sum_{k=0}^\infty c_k^2 = \|f\|^2.$

Для линейного пространства функций $H$, заданных таблично (в узлах),  скалярным произведением $(f,g) = \sum\limits_{i=0}^N f_i g_i$ и порождённой им нормой приближение строится аналогичным образом.

\subsection{Рассмотренные методы}