% starter for a character
\newpage
\phantomsection
\pagestyle{empty}
\vspace{0.5cm}
% \starter

\section*{Глава II. Интерполирование и приближение функций}

\subsection{Основные идеи главы} 

    Задача интерполирования состоит в нахождении значений функции $f(x)$
    на всем отрезке $[a,b]$ по ее значениям в узловых точках. $\forall f(x)~\exists!$ интерполяционный полином степени $n$, построенный по $(n+1)$-му узлу. Легко сроится в виде полинома Лагранжа или Ньютона.

Если требуется дополнительно наложить условия на производные порядка $(a_k-1)$, $k = \overline{0,m}$ и $(sum_{k=0}^m a_k = n + 1) \Rightarrow \exists!$ интерполяционный полином Эрмита степени $n$, удовлетворяющий значениям функции и её производных в заданных точках.

В гильбертовом пространстве $L_2$ вещественных функций, интегрируемых с квадратом ($\int\limits_a^b{f^2(x)dx} < \infty$) со скалярным произведением $(f,g)=\int\limits_a^b{f(x)g(x)dx}$ и порождённой им нормой, для заданной ф-ии $f$ и системы $(n+1)$ линейно независимых ф-й $\{\varphi_i(x)\}_{i=0}^n$ существует их линейная комбинация, имеющая минимальное отклонение по норме от функции $f(x)$. Если система ортонормированная –выполняется неравенство Бесселя $\sum_{k=0}^n c_k^2 \leqslant \|f\|^2$, для ортонормированного базиса достигается равенство Парсеваля $\sum_{k=0}^\infty c_k^2 = \|f\|^2.$

Для линейного пространства функций $H$, заданных таблично (в узлах),  скалярным произведением $(f,g) = \sum\limits_{i=0}^N f_i g_i$ и порождённой им нормой приближение строится аналогичным образом.

\newpage

\subsection{Рассмотренные методы}
\begin{adjustbox}{angle=90}
\begin{adjustbox}{scale=1.4}
\centering
\resizebox{\textwidth}{!}{
    \small
    \begin{tabular}{|p{3.5cm}|p{7cm}|p{7.1cm}|p{6cm}|}
    
    \hline
    \textbf{Название метода} & \textbf{Описание метода} & \textbf{Погрешность формулы}\\
    
    \hline
    \begin{tabular}[c]{@{}l@{}} Интерполяционный\\ полином Лагранжа \end{tabular}  &
    \begin{tabular}[c]{@{}l@{}}
    	\begin{math}L_n(x)=\sum_{k=0}^{n}{c_k(x)f(x_k)},\end{math}
        \\
        \begin{math}c_k(x)=\frac{\omega(x)}{(x-x_k)\omega'(x_k)},\end{math}
        \\
        \begin{math}\omega'(x_k)=\left(\prod_{\substack{i=0\\i\neq k}}^{n}{(x_k-x_i)}\right)\end{math} \\
    \end{tabular} &
    \begin{tabular}[c]{@{}l@{}}
    	\begin{math} \left|\psi_{L_n}(x)\right| \leqslant \frac{\sup_{x\in[a,b]}{\left|f^{(n+1)}
    (x)\right|}}
    {(n+1)!}\left|\omega(x)\right|,
   \end{math} \\
    \end{tabular}
    \\
    
    \hline
    \begin{tabular}[c]{@{}l@{}} Интерполяционный\\ полином Ньютона\end{tabular}  &
    \begin{tabular}[c]{@{}l@{}}
    	$N_n(x)=f(x_0)+(x-x_0)f(x_0,x_1)+$\\
        $+(x-x_0)(x-x_1)f(x_0,x_1,x_2)+\ldots+$\\
        $+(x-x_0)\ldots(x-x_{n-1})f(x_0,x_1,\ldots,x_n)$
        \\
        $f(x_0, x_{1}, \ldots, x_{k})=\sum_{i=0}^{k}{\frac{f(x_i)},{\omega_{0,k}'(x_i)}}$
    \end{tabular} &
    \begin{tabular}[c]{@{}l@{}}
    	см. полином Лагранжа, \\
        т.к. суть есть один многочлен
    \end{tabular}
    \\
    
    \hline
    \begin{tabular}[c]{@{}l@{}} Интерполяционный \\ полином Эрмита \end{tabular}  &
    \begin{tabular}[c]{@{}l@{}}
    	\begin{math}
        H_n(x)=\sum_{k=0}^m{\sum_{i=0}^{a_{k}-1}{c_{k,i}(x)f^{(i)}(x_k)}}
        \end{math}
        \\
        $c_{k,i}(x)$ - полиномы степени $n$.
        \\
        Коэффициенты ищутся из условия \\$H_n^{(i)}(x_k)=f^{(i)}(x_k)$
    \end{tabular} &
    \begin{tabular}[c]{@{}l@{}}
   	$\psi_{H_n}(x)=\frac{f^{(n+1)}(\xi)}{(n+1)!}(x-x_0)^{a_0}\ldots(x-x_m)^{a_m}$,\\
    $a_0+a_1+\ldots+a_m=n+1$
    \end{tabular}
    \\
    
     \hline
    \begin{tabular}[c]{@{}l@{}} Наилучшее\\среднеквадратичное \\приближение \end{tabular}  &
    \begin{tabular}[c]{@{}l@{}}
    	 $\overline{\varphi}(x) = \sum_{i=0}^n \overline{c_i} \varphi_i(x)$,\\
         коэффициенты ищутся из условия \\
         $\sum_{l=0}^n c_l(\varphi_k, \varphi_l) = (f, \varphi_k),\quad k = \overline{0,n},$\\
         $c_l, l = \overline{0,n}$\\
         \\
         (Аналогично для таблично \\заданных функций)
    \end{tabular} &
    \begin{tabular}[c]{@{}l@{}}
   	
    \end{tabular}
    \\
    
    \hline
    \end{tabular}
}
\end{adjustbox}
\end{adjustbox}