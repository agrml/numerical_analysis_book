% starter for a character
\newpage
\phantomsection
\pagestyle{empty}
\vspace{0.5cm}
% \starter

\section*{Глава VI. Приближённое вычисление определённых интегралов}

\subsection{Основные идеи главы} 
Одним из самых распространенных приемов приближенного вычисления определенных интегралов в вычислительной математике является подход, основанный на построении квадратурных формул. Смысл этих формул состоит в том, что определенный интеграл заменяется конечной суммой $\sum_{k=1}^{n}(c_k f(x_k))$, где $f(x_k)$ -- значение подынтегральной функции в узлах $x_k$, а выбор коэффициентов $c_k$ позволяет вычислить интеграл с требуемой точностью. Для рассмотренных в главе квадратурных форму исследуется погрешность вычисления как на частичном отрезке $[x_{i-1}, x_i], i = 1..n$, так и на всем промежутке интегрирования.

\begin{landscape}
\newpage
\subsection{Рассмотренные методы}


\end{landscape}
