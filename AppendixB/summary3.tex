% starter for a character
\newpage
\phantomsection
\pagestyle{empty}
\vspace{0.5cm}
% \starter


\section*{Глава III. Численное решение нелинейных уравнений и систем нелинейных уравнений}

\subsection{Основные идеи главы} 
 
Для решения нелинейных уравнений и систем нелинейных уравнений применяются методы последовательного приближения решения. Их идея заключается в задании некоторого начального приближения и последующим отдалением от него в поиске решений. 
 
В простейшем виде описанную идею реализует \textit{метод простой итерации}, где очередное приближение получается по формуле 
\begin{math} x^{n + 1} = x^n + r(x^n)f(x^n) \end{math}. 
Метод простой итерации не определяет функцию \begin{math} r() \end{math}. Оптимальным по скорости сходимости выбором является \begin{math} r(x) = -\frac{1}{f'(x^n)} \end{math}, что приводит к \textit{методу Ньютона}. 
% Недостатком метода Ньютона является высокая вычислительная сложность, исходящая из необходимости вычисления производной на каждой итерации. 
Выбор \begin{math} r(x) = - \frac{(x^n - x^{n - 1})f(x^n)}{f(x^n) - f(x^{n - 1})} \end{math} приводит к \textit{методу секущих}.

\newpage
\subsection{Рассмотренные методы}
\begin{adjustbox}{angle=90}
\begin{adjustbox}{scale=1.4}
\centering
\resizebox{\textwidth}{!}{
    \small
    \begin{tabular}{|p{3.5cm}|p{7cm}|p{7cm}|}
    \hline
    \textbf{Название метода} & \textbf{Формула прерощения} & \textbf{Условие сходимости} \\
    \hline
    \begin{tabular}[c]{@{}l@{}} Метод простой\\ итерации  \end{tabular}  &
    \begin{tabular}[c]{@{}l@{}}  \begin{math} \frac{x^{n + 1} - x^n}{\tau} + f(x^n) = 0, \end{math}\\ \begin{math} x^{n+1} = S(x), \end{math}\\ \begin{math} S(x) = x - \tau f(x), \end{math} \\ \begin{math} \tau > 0, n \in Z_+, x_0 \end{math} -- начальное\\ приближение, берется из \\ локализованной окрестности, \\ \begin{math} x_0 \in U_a(x^*) \end{math} \end{tabular} &
    \begin{tabular}[c]{@{}l@{}} Необходимое условие:\\ \begin{math} \sup_{x \in U_a(x^*)} |1 - \tau f'(x)| < 1, \end{math}\\т.е. \begin{math} 0< \tau < \frac{2}{M}, \end{math} \\ где \begin{math} M = \sup_{x \in U_a(x^*)} |f'(x)|  \end{math} \end{tabular}
        \\
    \hline
    \begin{tabular}[c]{@{}l@{}} Метод Ньютона \end{tabular} &
    \begin{tabular}[c]{@{}l@{}}  \begin{math} x^{n + 1} = x^n - \frac{f(x^n)}{f'(x^n)}, n = 0,1,... \end{math} \\ $x^0 \in U_a(x^*)$ \end{tabular} &
    \begin{tabular}[c]{@{}l@{}} $\exists M>0:  \frac{1}{2}\left|\left(\frac{f(x)f''(x)}{(f'(x))^2}\right)'\right| \leqslant M$,\\ $ \forall x \in U_a(x^*), |x^0 -x^*| \leqslant \frac{1}{M}$ \\ Тогда метод Ньютона сходится и \\ имеет место оценка \\ $|x^n - x^*| \leqslant \frac{1}{M} (M |x^0 - x^*|)^{2^n}$ \end{tabular}
        \\
    \hline
    \begin{tabular}[c]{@{}l@{}} Модифицированный \\ метод Ньютона  \end{tabular}&
    \begin{tabular}[c]{@{}l@{}} \begin{math} x^{n + 1} = x^n - \frac{f(x^n)}{f'(x^0)}n = 0,1,... \end{math} \\ $x^0 \in U_a(x^*)$   \end{tabular} & 
    \begin{tabular}[c]{@{}l@{}} Сходится быстрее метода простой \\ итерации,  но медленнее метода Ньютона  \end{tabular}\\
    \hline
    Метод секущих & 
    \begin{tabular}[c]{@{}l@{}} \begin{math} x^{n + 1} = - \frac{(x^n - x^{n - 1})f(x^n)}{f(x^n) - f(x^{n - 1})}, n = 0,1,...,\end{math}\\ $x^0$ и $x^1$ заданы \end{tabular} &
     \\
    \hline
    \end{tabular}
}
\end{adjustbox}
\end{adjustbox}