% starter for a character
\newpage
\phantomsection
\pagestyle{empty}
\vspace{0.5cm}
% \starter


\section{Глава I. Численные методы линейной алгебры}

\subsection{Основные идеи главы}

Для решения задач линейно алгебры, среди прочих, используются разнообразные вычислительные методы. Более того, некоторые задачи решаются исключительной вычислительными методами.

Методы решения делятся на прямые и итерационные. Прямые методы позволяют за конечное число шагов получить точное (с поправкой на ошибки округления) решение за конечное число шагов и оцениваются по числу умножений и делений. Для итерационных методов выбирается начальное приближение решения, которое на каждом шаге уточняется. Эффективность итерационного метода определяется числом операций, необходимых для получения решения с заданной точностью.

В главе I пособия рассматриваются основные методы решения систем линейных уравнений, поиска собственных значений матриц и нахождения обратной матрицы, а также освещается вопрос эффективности рассматриваемых методов.

нелинейных уравнений и систем нелинейных уравнений применяются методы последовательного приближения решения. Их идея заключается в задании некоторого начального приближения и последующим отдалением от него в поиске решений.

\subsection{Рассмотренные методы}
