% starter for a character
\newpage
\phantomsection
\pagestyle{empty}
\vspace{0.5cm}
% \starter


\section*{Глава I. Численные методы линейной алгебры}

\subsection{Основные идеи главы}

В главе I пособия рассматриваются основные методы решения систем линейных уравнений, поиска собственных значений матриц и нахождения обратной матрицы. Для численного решения СЛАУ применяются прямые и итерационные методы.

Прямые методы: например, метод Гаусса и квадратного корня — это методы, которые позволяют получать численные решения, исходя из формул, точно (с поправкой на ошибки округления). Эффективность прямых методов оценивается по числу действий (как правило умножений и делений), необходимых для реализации метода.

Итерационные методы основаны на том, что задается начальное приближение и указывается закон, по которому задаются следующие приближения. Эффективность итерационных методов оценивается числом итераций, необходимых для достижения заданной точности \begin{math} \epsilon \end{math}. Ясно, что чем меньшее число итераций потребуется, тем более эффективен метод.

Нахождение собственных значений матрицы, как правило, неразрешимая задача. В связи с этим в общем случае задача на поиск собственных значений решается численно с использованием итерационных методов. При численном решении задачи на собственные значения рассматриваются две проблемы: частичная проблема собственных значений, т.е. нахождение отдельных собственных значений и отвечающих им собственных векторов, и полной проблемы собственных значений, заключающася в нахождении всех собственных векторов матрицы.

Одно из важных понятий первой главы пособия — понятие обратной матрицы, которое активно используется не только в контексте прямого поиска решения, но и при исследовании на сходимость численных методов нахождения решений различных задач и оценке скорости их сходимости.

\subsection{Рассмотренные методы}
