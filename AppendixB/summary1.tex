% starter for a character
\newpage
\phantomsection
\pagestyle{empty}
\vspace{0.5cm}
% \starter


\section*{Глава I. Численные методы линейной алгебры}

\subsection{Основные идеи главы}

В главе I пособия рассматриваются основные методы решения систем линейных уравнений, поиска собственных значений матриц и нахождения обратной матрицы. Для численного решения СЛАУ применяются прямые и итерационные методы.

Прямые методы: например, метод Гаусса и квадратного корня — это методы, которые позволяют получать численные решения, исходя из формул, точно (с поправкой на ошибки округления). Эффективность прямых методов оценивается по числу действий (как правило умножений и делений), необходимых для реализации метода.

Итерационные методы основаны на том, что задается начальное приближение и указывается закон, по которому задаются следующие приближения. Эффективность итерационных методов оценивается числом итераций, необходимых для достижения заданной точности \begin{math} \varepsilon \end{math}. Ясно, что чем меньшее число итераций потребуется для нахождения численного решения с указанной выше точностью, тем более эффективен метод.

Нахождение собственных значений матрицы, как правило, аналитически неразрешимая задача. В связи с этим в общем случае задача на поиск собственных значений решается численно с использованием итерационных методов. При численном решении задачи на собственные значения рассматриваются две проблемы: частичная проблема собственных значений, т.е. нахождение отдельных собственных значений и отвечающих им собственных векторов, и полной проблемы собственных значений, состоящей в нахождении всех собственных векторов матрицы.

Одно из важных понятий первой главы пособия — понятие обратной матрицы, которое активно используется не только в контексте прямого поиска решения СЛАУ, но и при исследовании на сходимость численных методов решений различных задач и оценке скорости их сходимости.

\begin{landscape}
\newpage

\subsection{Рассмотренные методы}
\begin{adjustbox}{scale=0.75}
\centering
    \small
    \begin{tabular}{|p{3cm}|p{7cm}|p{6.5cm}|p{6.5cm}|}
    
    \hline
    \textbf{Название метода} & \textbf{Канонический вид} & \textbf{Условие сходимости в среднеквадратичной норме} & \textbf{Скорость сходимости}
    \\
    
    \hline
    \begin{tabular}[c]{@{}l@{}} Метод релаксации \end{tabular}  &
    \begin{tabular}[c]{@{}l@{}}
    	\begin{math} \frac{x^{n + 1} - x^n}{\tau} + Ax^n = f, \tau > 0, n \in Z_+ \end{math} \\
        \begin{math} x^0 \end{math} -- начальное приближение
    \end{tabular} &
    \begin{tabular}[c]{@{}l@{}}
    	\begin{math} A^*=A>0, \gamma=\max\limits_{1 \leq k \leq m}\lambda_k^A, 0<\tau<\frac{2}{\gamma} \end{math} \\
        \begin{math} x^0 \end{math} — любое
    \end{tabular} &
    \begin{tabular}[c]{@{}l@{}}
    	\begin{math} A=A^{*}>0,B=E \end{math} \\
        \begin{math} \gamma_1=\min\limits_{1 \leq k \leq m}\lambda_k^A, \gamma_2=\max\limits_{1 \leq k \leq m}\lambda_k^A \end{math} \\
        \begin{math} \xi=\frac{\gamma_1}{\gamma_2}, \rho=\frac{1-\xi}{1+\xi} \Rightarrow ||\nu^{n+1}|| \leq \rho||\nu^n|| \end{math}
    \end{tabular}
    \\
    
    \hline
    \begin{tabular}[c]{@{}l@{}} Метод Якоби \end{tabular}  &
    \begin{tabular}[c]{@{}l@{}}
    	\begin{math} D(x^{n + 1} - x^n) + Ax^n = f, n \in Z_+ \end{math} \\
        \begin{math} x^0 \end{math} -- начальное приближение \\
        \begin{math} D = diag(a_{11}, a_{22}, …, a_{mm}) \end{math}
    \end{tabular} &
    \begin{tabular}[c]{@{}l@{}}
    	Любое из: \\
    	1) \begin{math} A=A^{*}>0, 2D > A \end{math} \\
    	2) \begin{math} A=A^{*}>0, a_{ij} > \sum\limits_{j=1,i \neq j}^m |a_{ij}| \end{math} \\
        \begin{math} x^0 \end{math} — любое
    \end{tabular} &
    \begin{tabular}[c]{@{}l@{}}
    	\begin{math} A=A^{*}>0, B=B^{*}>0 \end{math} \\
        \begin{math} \exists \rho: 0 < \rho < 1, \frac{1-\rho}{\tau}B \leq A \leq \frac{1+\rho}{\tau}B \Rightarrow \end{math} \\
        \begin{math} \Rightarrow ||\nu^{n+1}||_B \leq \rho||\nu^n||_B \end{math}
    \end{tabular}
    \\
    
    \hline
    \begin{tabular}[c]{@{}l@{}} Метод Зейделя \end{tabular}  &
    \begin{tabular}[c]{@{}l@{}}
    	\begin{math} (D+R_1)(x^{n + 1} - x^n) + Ax^n = f \end{math} \\
        \begin{math} n \in Z_+, x^0 \end{math} — нач. приближение \\
        \begin{math} D = diag(a_{11}, a_{22}, …, a_{mm}) \end{math} \\
        \begin{math} R_1 = 
        	\begin{bmatrix}
                0      & 0      & 0      & \dots  & 0      \\
                a_{21} & 0      & 0      & \dots  & 0      \\
                \vdots & \vdots & \vdots & \ddots & \vdots \\
                a_{m1} & a_{m2} & a_{m3} & \dots  & 0
            \end{bmatrix}
        \end{math}
    \end{tabular} &
    \begin{tabular}[c]{@{}l@{}}
    	\begin{math} A = A^{*}>0, x^0 \end{math} — любое
    \end{tabular} &
    \begin{tabular}[c]{@{}l@{}}
    	\begin{math} A=A^{*}>0, B=B^{*}>0 \end{math} \\
        \begin{math} \exists \gamma_1, \gamma_2: 0 < \gamma_1 < \gamma_2, \gamma_1B \leq A \leq \gamma_2B \end{math} \\
        \begin{math} \tau = \frac{2}{\gamma_1+\gamma_2}=\tau_0 \end{math} \\
        \begin{math} \xi=\frac{\gamma_1}{\gamma_2}, \rho=\frac{1-\xi}{1+\xi} \Rightarrow ||\nu^{n+1}||_B \leq \rho||\nu^n||_B \end{math}
    \end{tabular}
    \\
    
    \hline
    \begin{tabular}[c]{@{}l@{}} Попеременно- \\ треугольный \\ итерационный \\ метод \end{tabular}  &
    \begin{tabular}[c]{@{}l@{}}
    	\begin{math} (E+\omega R_1)(E+\omega R_2)\frac{x^{n + 1} - x^n}{\tau} + Ax^n = f \end{math} \\
        \begin{math} n \in Z_+, \omega > 0, \tau > 0 \end{math} \\
        \begin{math} x^0 \end{math} — начальное приближение \\
        \begin{math} R_1 = 
        	\begin{bmatrix}
                \frac{a_{11}}{2} & 0                & 0      & \dots  & 0                \\
                a_{21}           & \frac{a_{22}}{2} & 0      & \dots  & 0                \\
                \vdots           & \vdots           & \vdots & \ddots & \vdots           \\
                a_{m1}           & a_{m2}           & a_{m3} & \dots  & \frac{a_{mm}}{2}
            \end{bmatrix}
        \end{math} \\
        \begin{math} R_2 = 
        	\begin{bmatrix}
                \frac{a_{11}}{2} & a_{12}           & a_{13} & \dots  & a_{1m}           \\
                0                & \frac{a_{22}}{2} & 0      & \dots  & a_{2m}           \\
                \vdots           & \vdots           & \vdots & \ddots & \vdots           \\
                0                & 0                & 0      & \dots  & \frac{a_{mm}}{2}
            \end{bmatrix}
        \end{math}
    \end{tabular} &
    \begin{tabular}[c]{@{}l@{}}
    	\begin{math} A = A^{*}>0, \omega > \frac{\tau}{4}, x^0 \end{math} — любое
    \end{tabular} &
    \begin{tabular}[c]{@{}l@{}}
    	\begin{math} A=A^{*}>0, \exists\delta, \Delta>0: A > \delta E \end{math} \\
        \begin{math} R_2^*R_2\leq\frac{\Delta}{4}A \end{math} \\
        \begin{math} \gamma_1=\frac{\sqrt[]{\delta}\sqrt[]{\delta\Delta}}{2(\sqrt[]{\delta}+\sqrt[]{\Delta})}, \gamma_2=\frac{\sqrt[]{\delta\Delta}}{4}, \end{math} \\
        \begin{math} \omega=\frac{2}{\sqrt[]{\delta\Delta}}, \tau=\frac{2}{\gamma_1+\gamma_2} \end{math} \\
        \begin{math} \eta=\frac{\delta}{\Delta}, \rho=\frac{1-\sqrt[]{\eta}}{1+3\sqrt[]{\eta}} \end{math} \\
        \begin{math} B=(E+\omega R_1)(E+\omega R_2) \Rightarrow \end{math} \\
        \begin{math} \Rightarrow ||\nu^{n+1}||_B \leq \rho||\nu^n||_B \end{math}
    \end{tabular}
    \\
    
    \hline
    \end{tabular}
\end{adjustbox}
\end{landscape}
